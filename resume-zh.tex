%%
%% Copyright (c) 2018-2019 Weitian LI <wt@liwt.net>
%% CC BY 4.0 License
%%
%% Created: 2018-04-11
%%

\hbadness=99999 

% Chinese version
\documentclass[zh]{resume}

% Adjust icon size (default: same size as the text)
\iconsize{\Large}

% File information shown at the footer of the last page
\fileinfo{%
  \githublink{Oyami-Srk}{Resume} \hspace{0.5em}
  \faEdit{} \today
}
\headercontent{本简历由Github Action自动构建,请访问\link{https://blog.qvq.moe/Resume/resume-zh.pdf}{本链接}查看最新版本}

\name{昊轩}{韩}

\keywords{Rust, Linux, Programming, Python, C, Shell, Bare matel}

% \tagline{\icon{\faBinoculars}} <position-to-look-for>}
% \tagline{<current-position>}

% \photo{<height>}{<filename>}

\profile{
  \mobile{176-6409-4001}
  \email{hhx.xxm@gmail.com}
  \github{Oyami-Srk} \\
  \university{齐鲁理工学院}
  \degree{计算机科学与技术 \textbullet 学士(2024届)}
  \birthday{2001-01-25}
  \address{青岛}
  % Custom information:
  % \icontext{<icon>}{<text>}
  % \iconlink{<icon>}{<link>}{<text>}
}

\begin{document}
\makeheader

%======================================================================
% Summary & Objectives
%======================================================================
\begin{paragraph}
对计算机底层、操作系统等有深入的研究。热衷于操作系统技术,编写过 x86、RISC-V 架构的操作系统内核。熟悉Linux内核和GNU/Linux系统,并对嵌入式软件开发有一定经验。 
\end{paragraph}
\begin{paragraph}
有充分的计算机底层、裸机开发的经验,能够独立完成操作系统内核级别的设计、开发与调试等。对各类开发环境和工具链有丰富的使用经验,擅长快速学习新技术。实际应用能力突出,能够完成全栈工程师的工作。
\end{paragraph}
\begin{paragraph}
热爱开源社区,在\link{https://github.com/Oyami-Srk}{GitHub} 上开源了多个个人项目,并对一些开源项目做出过贡献。在校期间曾担任计算机协会会长,荣获优秀学生称号,并多次获得校奖学金。
\end{paragraph}


\sectionTitle{个人技能}{\faWrench}
\begin{competences}[7em]
  \comptence{程序设计语言}{
	\textbf{C}\,\tag{精通}、\textbf{Python}\,\tag{精通}\,、\textbf{TypeScript}\,\tag{熟练}\,、\textbf{Rust}\,\tag{熟练}\,、\textbf{Shell}\,\tag{熟练}\,
  }
  \comptence{开发工具和框架}{
	\textbf{Git}、\textbf{Vue}、\textbf{Django}、\textbf{FreeRTOS}
  }
  \comptence{\icon{\faLanguage} 语言}{
    \textbf{英语(CET4 604分)} --- 习惯使用英文进行技术资料的阅读、编写等。
  }
\end{competences}

% %======================================================================
% \sectionTitle{教育背景}{\faGraduationCap}
% %======================================================================
% \begin{educations}
%   \education%
%     {2020.09}%
%     [2024.06]%
%     {齐鲁理工学院}%
%     {计算机与信息工程学院}%
%     {计算机科学与技术}%
%     {学士(在读)}
% \end{educations}

%======================================================================
%\sectionTitle{计算机技能}{\faCogs}
%======================================================================
%\begin{itemize}
%	\item 熟练掌握以下编程语言和技术:C, Shell, Python, Rust, Typescript。
%	\item 能够熟练使用英文,具有优秀的读写能力,能够使用英文进行日常交流、阅读文档。
%	% \item 优秀的学习能力,热爱新技术,自学能力强。
%	\item 有充分的计算机底层、裸机开发的经验,能够独立完成操作系统内核级别的设计与开发。%,对Linux系统内核原理有深入的了解。
%	\item 具备一定的嵌入式软件开发能力,可以开发出稳定的嵌入式软件。
%	\item 熟练使用Linux操作系统,具备开发基础工具的能力,可以根据需求编写特定功能的脚本、程序等。
%	\item 熟练使用Git、Shell、CMake等工具,对主流的集成开发环境有丰富的使用经验。
%	% \item 能够快速上手新技术、新框架、新语言,可以对特定需求进行快速开发。
%\end{itemize}

%======================================================================
\sectionTitle{实习经历}{\faCode}
%======================================================================
\begin{experiences}
  \experience%
    [2023.09]%
    {2023.12}%
    {技术支持工程师 @ \textbf{统信软件技术有限公司(深圳分公司)}}%
    [\begin{itemize}
		\item 负责Linux开源软件的打包及维护工作
		\item 使用爬虫技术自动化工作流程,并参与部门的技术支持工作。
		\item 独立研发\link{https://github.com/shirodeb}{ShiroDEB}工具集,并基于此编写和维护自动构建脚本。
		\item 参加公司对外主办的技术分享活动,分享技术心得、介绍工作成果(2023-11 武汉LUG)。
		\item 使用Docker技术及Github Action持续集成ShiroDEB工作环境,创建构建净室环境。
    \end{itemize}]
\end{experiences}

%======================================================================
\sectionTitle{个人项目}{\faCode}
%======================================================================
\begin {projects}
	\project{2023.8}[2023.12]{安规培训报名信息平台及周边设施}{
		个人外包项目,于实习期业余时间编写维护。本项目实现了一个安规培训报名信息、考试信息管理平台,并支持Excel数据表的导入导出功能。目前正在维护期。
		\begin{itemize}
			\item 使用Django及Django REST Framework作为后端实现
			\item 使用Vue3.0、Electron进行管理系统的开发,使用Svelte进行学员报名页的开发
			\item 基于ID100身份证读取器和Java FX实现了一个与现行系统交互的考试人脸比对签到软件。
			\item 使用Docker和Docker-Compose实现容器化,提高部署效率。
		\end{itemize}
	}[全栈开发,Docker,运维]

	\project{2023.7}[2023.10][https://gitee.com/BookOS/nxos/tree/proj-3-virtio/]{基于NXOS的VirtIO驱动实现}{
		\link{https://summer-ospp.ac.cn/org/prodetail/238070164?lang=zh&list=pro}{开源之夏2023}项目,为NXOS内核实现VirtIO系列驱动。
		\begin{itemize}
			\item 实现了VirtIO Over PCI Bus及VirtIO Over MMIO通讯方式
			\item 实现了常见的VirtIO设备,例如Net、Block、Input、Sound等
			\item 开发过程中修复了该内核的代码错误
			\item 开发过程中通过阅读Linux及QEMU的源码对不满足规范的错误实现进行兼容
		\end{itemize}
	}[C,VirtIO,QEMU,底层开发,驱动开发]

	\project{2023.4}[2023.7]{在线安规培训平台}{
		个人外包项目。本项目实现了一个在线视频培训平台,具有视频学习、人脸检测、试题测验、留言评论等功能。
		\begin{itemize}
			\item 使用Django及Django REST Framework作为后端实现
			\item 使用Vue3.0及Vue-Pure-Admin框架进行后台管理人员界面的开发
			\item 分别使用Vue及微信小程序开发了具备同样功能的学习者前端使用界面
			\item 使用Docker和Docker-Compose完成容器化,提高部署效率
		\end{itemize}
	}[Python,Django,Django REST Framework,Vue,全栈开发,Docker]

	\project{2022.5}[2022.7][https://github.com/Oyami-Srk/MFTT-RISCV]{基于RISC-V架构的操作系统内核}{
		\link{https://os.educg.net/\#/oldDetail?name=2022\%E5\%85\%A8\%E5\%9B\%BD\%E5\%A4\%A7\%E5\%AD\%A6\%E7\%94\%9F\%E8\%AE\%A1\%E7\%AE\%97\%E6\%9C\%BA\%E7\%B3\%BB\%E7\%BB\%9F\%E8\%83\%BD\%E5\%8A\%9B\%E5\%A4\%A7\%E8\%B5\%9B\%E6\%93\%8D\%E4\%BD\%9C\%E7\%B3\%BB\%E7\%BB\%9F\%E8\%AE\%BE\%E8\%AE\%A1\%E8\%B5\%9B-\%E5\%86\%85\%E6\%A0\%B8\%E5\%AE\%9E\%E7\%8E\%B0\%E8\%B5\%9B}{全国大学生操作系统设计大赛}作品,基于RISC-V 64架构的操作系统内核,使用宏内核架构。通过调用SBI实现内核与硬件平台的交互,具备可移植性和通用性。
		\begin{itemize}
			\item 通过自旋锁及睡眠锁实现了对称多处理的支持
			\item 实现了基于伙伴算法的页分配器及基于红黑树的Slab对象分配器
			\item 实现了CoW Fork,优化系统性能
			\item 实现了扁平设备树文件的支持,并带有可拓展的驱动框架
			\item 实现了Execve、dup、pipe等常用的POSIX系统调用,并兼容部分Linux系统调用
			\item 内核支持QEMU和K210,并针对QEMU实现了virtio-mmio及virtio-disk驱动
			\item 在开发本项目时,同时开发了\link{https://github.com/Oyami-Srk/RISCV-GDB-Paging} {\texttt{RISCV-GDB-Paging}}用于调试RISC-V架构SV39/SV48分页信息的GDB脚本,通过Python及Scheme Lisp实现,该脚本能令调试QEMU平台下RISC-V架构的分页信息变得简单易懂。
		\end{itemize}
	}[C,CMake,底层开发,内核开发,RISC-V,GDB,Python]

	\project{2021.10}[2022.3][https://github.com/Oyami-Srk/SelfServiceCarWashing]{自助式洗车机嵌入式软件}{
		校企合作项目。本项目为自助式洗车机解决方案中的嵌入式软件部分,本项目实现了终端电气控制及用户使用界面。
		\begin{itemize}
			\item 本人作为项目管理者负责协调不同方向的开发人员,同时与合作企业进行交流协作
			\item 本项目采用STM32F4系列主控,并基于HAL库进行二次开发
			\item 实现了LVGL的基于DMA2D的高性能图形操作的移植,并将LVGL GUI程序与嵌入式功能解耦,便于调试和编写
			\item 通过AT协议实现了与LTE物联网模块和ESP32的对接,并基于此与服务器端进行交流
			\item 实现了通过USB-FS及FATFS对配置和资源文件的修改
			\item 本项目采用FreeRTOS作为嵌入式系统,并使用多个任务完成不同的功能部分
		\end{itemize}
	}[C,STM32,LVGL,FreeRTOS,嵌入式开发]
\end {projects}

\sectionTitle{其他个人项目}{\faCode}
\begin{itemize}
	\item \simpleProject[https://github.com/Oyami-Srk/OmochaOS/tree/backup-main-cmake-2022]{OmochaOS}{
		个人学习用x86系统内核,采用微内核架构。实现了HPET、PCI、APIC、AHCI等驱动程序,并具备可拓展的系统模块加载框架。
	}[C,CMake,内核开发,底层开发]
	\item \simpleProject[https://github.com/Oyami-Srk/AiR]{AiR空气质量监测器}{
		AiR空气质量监测器系统。基于ESP32。项目实现了MQTT、HTTP API、HTTP界面等多种网络信息传递方式。同时也能于显示屏上展示各种信息。本项目取得了软件著作权证书。
	}[C,嵌入式开发,FreeRTOS,ESP32]
	\item \simpleProject[https://github.com/Oyami-Srk/OmegaGomoku]{OmegaGomoku}{
		本项目为基于PyTorch和DQN算法的五子棋人工智能。项目总结了一些DQN算法的实现,优化了传统DQN的奖励计算过程。
		本项目在10万轮的训练中表现出一定水平的智能,与基于MiniMax算法的传统搜索算法相比,在两层的搜索深度下最高能达到95\%的胜率,平均胜率则超过50\%。
		项目报告:\link{https://blog.qvq.moe/OmegaGomoku/Report/\%E4\%BA\%BA\%E5\%B7\%A5\%E6\%99\%BA\%E8\%83\%BD\%E8\%AF\%BE\%E7\%A8\%8B\%E6\%8A\%A5\%E5\%91\%8A.pdf}{点击查看项目报告}
	}[Python,PyTorch,DQN,机器学习]
	\item \simpleProject[https://github.com/Oyami-Srk/Rust-shunting\_yard]{Rust-shunting\_yard}{
		使用Rust编写的基于调度场算法的表达式求值工具,具备一定的函数定义能力。
	}[Rust,Algorithm]
	\item \simpleProject[https://github.com/Oyami-Srk/rust-headless-chrome/tree/resp-handler-dereg]{Rust-headless-chrome(代码贡献)}{
		对广泛应用的Rust的Chrome DevTools协议实现API提交了代码贡献,完善了部分API的应用方法,并针对其代码生成工具\link{https://github.com/Oyami-Srk/auto\_generate\_cdp}{\texttt{auto\_generate\_cdp}}中的错误进行修正。
	}[Rust,Library,Chrome DevTools]
	\item \simpleProject[https://github.com/Oyami-Srk/shirodl]{ShiroDL}{
		Rust异步并发小文件下载库,具有良好的通用性和拓展性。同时提供批量下载小文件的命令行程序。
	}[Rust,Library,工具开发]
\end{itemize}

%======================================================================
\sectionTitle{奖项证书}{\faTrophy}
%======================================================================
\begin{itemize}
  	\item\link{https://qvq.moe:5001/public/imgs/2018-GoogleCodeIn\%E9\%AB\%98\%E4\%B8\%AD.jpg}{\texttt{Google Code-in 2017}}
	\item\link{https://qvq.moe:5001/public/imgs/2020-\%E5\%A4\%96\%E7\%A0\%94\%E7\%A4\%BE\%E5\%9B\%BD\%E6\%9D\%90\%E6\%9D\%AF\%E5\%85\%A8\%E5\%9B\%BD\%E8\%8B\%B1\%E8\%AF\%AD\%E9\%98\%85\%E8\%AF\%BB\%E5\%A4\%A7\%E8\%B5\%9B\%E5\%B1\%B1\%E4\%B8\%9C\%E7\%9C\%81\%E5\%A4\%8D\%E8\%B5\%9B-\%E4\%B8\%89\%E7\%AD\%89\%E5\%A5\%96.jpg} {\texttt{2020\enquote{外研社·国才杯}全国英语阅读大赛\ 山东赛区\ 三等奖}}
	\item\link{https://qvq.moe:5001/public/imgs/2021-\%E4\%B8\%AD\%E5\%9B\%BD\%E5\%A4\%A7\%E5\%AD\%A6\%E7\%94\%9F\%E8\%AE\%A1\%E7\%AE\%97\%E6\%9C\%BA\%E8\%AE\%BE\%E8\%AE\%A1\%E5\%A4\%A7\%E8\%B5\%9B\%E5\%B1\%B1\%E4\%B8\%9C\%E7\%9C\%81\%E7\%BA\%A7\%E8\%B5\%9B-\%E4\%B8\%89\%E7\%AD\%89\%E5\%A5\%96.jpg} {\texttt{2021年第14届中国大学生计算机设计大赛\ 山东省级赛\ 三等奖}}
	\item\link{https://qvq.moe:5001/public/imgs/2021-\%E5\%85\%A8\%E5\%9B\%BD\%E5\%A4\%A7\%E5\%AD\%A6\%E7\%94\%9F\%E8\%8B\%B1\%E8\%AF\%AD\%E7\%AB\%9E\%E8\%B5\%9BC\%E7\%B1\%BB-\%E4\%BA\%8C\%E7\%AD\%89\%E5\%A5\%96.jpg} {\texttt{2021年第三届全国高校计算机能力挑战赛\ 程序设计赛\,C++华东区域\ 优秀奖}}
	\item\link{https://qvq.moe:5001/public/imgs/2021-\%E5\%B1\%B1\%E4\%B8\%9C\%E7\%9C\%81\%E5\%A4\%A7\%E5\%AD\%A6\%E7\%94\%9F\%E7\%94\%B5\%E5\%AD\%90\%E4\%B8\%8E\%E4\%BF\%A1\%E6\%81\%AF\%E6\%8A\%80\%E6\%9C\%AF\%E5\%BA\%94\%E7\%94\%A8\%E5\%A4\%A7\%E8\%B5\%9B-\%E4\%BA\%8C\%E7\%AD\%89\%E5\%A5\%96.jpg} {\texttt{\enquote{赛冠杯}第八届山东省大学生电子与信息技术应用大赛\ 二等奖}}
	\item\link{https://qvq.moe:5001/public/imgs/2022-\%E4\%B8\%AD\%E5\%9B\%BD\%E5\%A4\%A7\%E5\%AD\%A6\%E7\%94\%9F\%E8\%AE\%A1\%E7\%AE\%97\%E6\%9C\%BA\%E8\%AE\%BE\%E8\%AE\%A1\%E5\%A4\%A7\%E8\%B5\%9B\%E7\%9C\%81\%E4\%BA\%8C\%E7\%AD\%89\%E5\%A5\%96.jpg} {\texttt{2022年第15届中国大学生计算机设计大赛\ 山东省级赛\ 二等奖}}
	\item\link{https://qvq.moe:5001/public/imgs/2022-\%E6\%93\%8D\%E4\%BD\%9C\%E7\%B3\%BB\%E7\%BB\%9F\%E8\%AE\%BE\%E8\%AE\%A1\%E5\%88\%9D\%E8\%B5\%9B\%E4\%BC\%98\%E8\%83\%9C\%E5\%A5\%96.jpg} {\texttt{2022年全国大学生计算机系统能力大赛操作系统设计赛内核实现赛道\ 初赛优胜奖}}
	\item\link{https://qvq.moe:5001/public/imgs/2022-\%E8\%93\%9D\%E6\%A1\%A5\%E6\%9D\%AF\%E5\%9B\%BD\%E8\%B5\%9BB\%E7\%BB\%84-\%E4\%B8\%89\%E7\%AD\%89\%E5\%A5\%96.jpg} {\texttt{第十三届蓝桥杯全国软件和信息技术专业人才大赛\ 山东赛区C/C++程序设计\ 大学B组\ 一等奖}}
	\item\link{https://qvq.moe:5001/public/imgs/2022-\%E8\%93\%9D\%E6\%A1\%A5\%E6\%9D\%AF\%E5\%9B\%BD\%E8\%B5\%9BB\%E7\%BB\%84-\%E4\%B8\%89\%E7\%AD\%89\%E5\%A5\%96.jpg} {\texttt{第十三届蓝桥杯全国软件和信息技术专业人才大赛\ 全国总决赛C/C++程序设计\ 大学B组\ 三等奖}}
	\item\link{https://qvq.moe:5001/public/imgs/\%E8\%BD\%AF\%E4\%BB\%B6\%E8\%91\%97\%E4\%BD\%9C\%E6\%9D\%83\%E7\%99\%BB\%E8\%AE\%B0\%E8\%AF\%81\%E4\%B9\%A6-\%E7\%A9\%BA\%E6\%B0\%94\%E8\%B4\%A8\%E9\%87\%8F\%E6\%A3\%80\%E6\%B5\%8B\%E7\%B3\%BB\%E7\%BB\%9F.jpg} {\texttt{中华人民共和国国家版权局\ 计算机软件著作权登记证书\ AiR空气质量监测器系统}}
\end{itemize}


\end{document}
