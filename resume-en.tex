%%
%% Copyright (c) 2018-2019 Weitian LI <wt@liwt.net>
%% CC BY 4.0 License
%%
%% Created: 2018-04-11
%%

\hbadness=99999 

% Chinese version
\documentclass[en]{resume}

% Adjust icon size (default: same size as the text)
\iconsize{\Large}

% File information shown at the footer of the last page
\fileinfo{%
  \githublink{Oyami-Srk}{Resume} \hspace{0.5em}
  \faEdit{} \today
}
\headercontent{This resume is automatically generated by GitHub Action, Please visit \link{https://blog.qvq.moe/Resume/resume-en.pdf}{this link} for the latest version.}

\name{Haoxuan}{Han}

\keywords{Rust, Linux, Programming, Python, C, Shell, Bare matel}

% \tagline{\icon{\faBinoculars}} <position-to-look-for>}
% \tagline{<current-position>}

% \photo{<height>}{<filename>}

\profile{
  \mobile{176-6409-4001}
  \email{hhx.xxm@gmail.com}
  \github{Oyami-Srk} \\
  \university{Qilu Insitute of Technology}
  \degree{Computer Science \textbullet Bachelor}
  \birthday{2001-01-25}
  \address{Qingdao}
  % Custom information:
  % \icontext{<icon>}{<text>}
  % \iconlink{<icon>}{<link>}{<text>}
}

\begin{document}
\makeheader

%======================================================================
% Summary & Objectives
%======================================================================
{\onehalfspacing
\hspace{2em}
Love and understand low-level and kernel development, Passionate about operating system technology.
Developed kernels for x86 and RISC-V architectures. Familiar with Linux kernel and GNU/Linux. Experienced in embedded software development.
With extensive experience in bare-metal development, and be capable of designing, implementing and debugging large projects independently. Knowing many popular development environments and toolchains. Good at learning new technologies rapidly.
Outstanding practical application ability, capable of performing full-stack engineering work.
Enjoy the open-source community, have open-sourced various personal projects on \link{https://github.com/Oyami-Srk}{GitHub} and contributed to some open-source projects. 
During school, served as the president of the Computer Association, won the title of outstanding student, and received scholarships multiple times.
\par}


\sectionTitle{Skills}{\faWrench}
\begin{competences}[11em]
  \comptence{Prog. Lang.}{
	\textbf{C}\,\tag{Excellent}, \textbf{Python}\,\tag{Excellent}\,, \textbf{TypeScript}\,\tag{Proficient}\,, \textbf{Rust}\,\tag{Proficient}\,, \textbf{Shell}\,\tag{Excellent}\,
  }
  \comptence{Tools \& Frameworks}{
	\textbf{Git}, \textbf{Vue}, \textbf{Django}, \textbf{FreeRTOS}
  }
  \comptence{\icon{\faLanguage} Language}{
    \textbf{English (CET4 604)} --- Preferred reading and writing documents in English.
  }
\end{competences}

% %======================================================================
% \sectionTitle{教育背景}{\faGraduationCap}
% %======================================================================
% \begin{educations}
%   \education%
%     {2020.09}%
%     [2024.06]%
%     {齐鲁理工学院}%
%     {计算机与信息工程学院}%
%     {计算机科学与技术}%
%     {学士(在读)}
% \end{educations}

%======================================================================
%\sectionTitle{计算机技能}{\faCogs}
%======================================================================
%\begin{itemize}
%	\item 熟练掌握以下编程语言和技术:C, Shell, Python, Rust, Typescript。
%	\item 能够熟练使用英文,具有优秀的读写能力,能够使用英文进行日常交流、阅读文档。
%	% \item 优秀的学习能力,热爱新技术,自学能力强。
%	\item 有充分的计算机底层、裸机开发的经验,能够独立完成操作系统内核级别的设计与开发。%,对Linux系统内核原理有深入的了解。
%	\item 具备一定的嵌入式软件开发能力,可以开发出稳定的嵌入式软件。
%	\item 熟练使用Linux操作系统,具备开发基础工具的能力,可以根据需求编写特定功能的脚本、程序等。
%	\item 熟练使用Git、Shell、CMake等工具,对主流的集成开发环境有丰富的使用经验。
%	% \item 能够快速上手新技术、新框架、新语言,可以对特定需求进行快速开发。
%\end{itemize}

%======================================================================
\sectionTitle{Internship}{\faCode}
%======================================================================
\begin{experiences}
  \experience%
    [2023.09]%
    {2023.12}%
    {Technical Support Engineer @ \textbf{UnionTech Software Technology Co., Ltd. (Shenzhen Branch)}}%
    [\begin{itemize}
		\item Packaging and maintenance of Linux open-source software.
		\item Wrote scripts for workflow automation and spider program. Participated in the department's technical support work.
		\item Independently developed the \link{https://github.com/shirodeb}{ShiroDEB} scripts set and maintained automatic build scripts based on this.
		\item Participated in the company-hosted external technical sharing activities, shared technical insights, and introduced work results (2023-11 Wuhan LUG).
		\item Used Docker technology and Github Action to continuously integrate the ShiroDEB working environment, creating the clean room build environment.
    \end{itemize}]
\end{experiences}


%======================================================================
\sectionTitle{Personal Projects}{\faCode}
%======================================================================
\begin {projects}
	\project{2023.8}[2023.12]{Safety Training Information Platform and Surrounding Facilities}{
		Personal freelance project. Written and maintained in my spare time during the internship. Implemented a safety training signup information records and exam information management platform, and supports the import and export functions of Excel files. It is currently in the maintenance period.
		\begin{itemize}
			\item Use Django and Django REST Framework for backend implementation
			\item Use Vue3.0, Electron for the management system, and use Svelte for the student registration page.
			\item Implemented exam check-in software based on the ID100 ID card reader and Java FX. This software communicates with both this management platform and the Government-designated examination management system.
			\item Use Docker and Docker-Compose to implement containerization, improving deployment efficiency.
		\end{itemize}
	}[Full-stack,Docker]

	\project{2023.7}[2023.10][https://gitee.com/BookOS/nxos/tree/proj-3-virtio/]{VirtIO Drivers implementation based on NXOS Kernel}{
		Project of \link{https://summer-ospp.ac.cn/org/prodetail/238070164?lang=zh&list=pro}{OSPP 2023} (an event like GSOC), implemented VirtIO Drivers Framework and frontend devices drivers for NXOS Kernel. Project mentors expressed high approval of this project.
		\begin{itemize}
			\item Implemented VirtIO Over PCI Bus and VirtIO Over MMIO
			\item Implemented various VirtIO devices, such as Net, Block, Input, Sound, etc.
			\item Implemented generic framework for any other VirtIO devices to be implemented in the future
			\item Fixed Bugs in NXOS Kernel codebase
			\item Archived compatibility to error implementations beyond specification through studying the sources of Linux kernel and QEMU
		\end{itemize}
	}[C,VirtIO,QEMU,Low-level,Driver]

	\project{2023.4}[2023.7]{Safety Training Online Platform}{
		Personal freelance project. Implemented an online video training platform with features of online training, face recognition, quiz tests and comments.
		\begin{itemize}
			\item Using Django and Django REST Framework for backend implementation
			\item Using Vue3.0 and Vue-Pure-Admin framework for administration panel
			\item Implemented the same functionality by Vue and WeChat mini program for trainee frontend
			\item Containerize the whole project with Docker and Docker-Compose to improve the deployment efficiency
		\end{itemize}
	}[Python,Django,Django REST Framework,Vue,Full-stack,Docker]

	\project{2022.5}[2022.7][https://github.com/Oyami-Srk/MFTT-RISCV]{OS Kernel on RISC-V Architecture}{
		Project for \link{https://os.educg.net/\#/oldDetail?name=2022\%E5\%85\%A8\%E5\%9B\%BD\%E5\%A4\%A7\%E5\%AD\%A6\%E7\%94\%9F\%E8\%AE\%A1\%E7\%AE\%97\%E6\%9C\%BA\%E7\%B3\%BB\%E7\%BB\%9F\%E8\%83\%BD\%E5\%8A\%9B\%E5\%A4\%A7\%E8\%B5\%9B\%E6\%93\%8D\%E4\%BD\%9C\%E7\%B3\%BB\%E7\%BB\%9F\%E8\%AE\%BE\%E8\%AE\%A1\%E8\%B5\%9B-\%E5\%86\%85\%E6\%A0\%B8\%E5\%AE\%9E\%E7\%8E\%B0\%E8\%B5\%9B}{Kernel Design Competition to University Student}, a macro-kernel based on RISC-V 64 architecture. Communicate with hardware via SBI and therefore the portability and universality could be archived.
		\begin{itemize}
			\item Implemented SMP multi-processor via spinlock and sleeplock
			\item Implemented Buddy Page Allocator and RB-Tree based Slab objects allocator
			\item Implemented CoW Fork to improve performance
			\item Implemented the support of Flatten Device Tree along with extensible driver framework
			\item Implemented some generic POSIX syscalls like execve, dup and pipe and some Linux syscalls
			\item Support QEMU and K210, and implemented virtio-mmio/virtio-disk for QEMU
			\item Developed \link{https://github.com/Oyami-Srk/RISCV-GDB-Paging} {\texttt{RISCV-GDB-Paging}} for debugging SV39/SV48 paging information for RISC-V on QEMU. Written in Python and Scheme Lisp
		\end{itemize}
	}[C,CMake,Low-level,Kenrel,RISC-V,GDB,Python]

	\project{2021.10}[2022.3][https://github.com/Oyami-Srk/SelfServiceCarWashing]{Self Service Car Washing Solution, Embedded Software}{
		Embedded part of Self Service Car Washing solutions, this project implemented the terminal electrical controlling and user-end GUI.
		\begin{itemize}
			\item As the project manager of the whole team, coordinated developers in other parts and collaborated with partner company
			\item Using STM32F4 as MCU and developed based on STM32 HAL
			\item Ported LVGL using DMA2D for higher performance, decoupling the LVGL GUI from embedded functionality for easier debugging and development
			\item Communicate with LTE IoT modules and ESP32 via AT commands, enabling interaction with server-side systems
			\item Implemented the configurating from PC via USB-FS and FATFS
			\item Use FreeRTOS as the embedded operating system, using tasks to handle different sub-procedures
		\end{itemize}
	}[C,STM32,LVGL,FreeRTOS,Embedded]
\end {projects}

\sectionTitle{Other Personal Projects}{\faCode}
\begin{itemize}
	\item \simpleProject[https://github.com/Oyami-Srk/OmochaOS/tree/backup-main-cmake-2022]{OmochaOS}{
		Toy kernel for study of x86 development, using micro-kernel architecture. Implemented drivers for essential components such as HPET, PCI, APIC, and AHCI. Additionally, a modular framework to ensure extensibility and flexibility within the system has been implemented.
	}[C,CMake,Kernel development,Low-level developement]
	\item \simpleProject[https://github.com/Oyami-Srk/AiR]{AiR Air Quality Monitoring System}{
		The AiR Air Quality Monitoring System based on ESP32. With MQTT, HTTP API and HTTP Frontend support, also with display support on onboard LCD.
	}[C,Embedded,FreeRTOS,ESP32]
	\item \simpleProject[https://github.com/Oyami-Srk/OmegaGomoku]{OmegaGomoku}{
		A Gomoku AI using PyTorch and the DQN algorithm. Optimized DQN reward calculation. Achieved moderate AI performance after 100k training games. Outperformed MiniMax with 95\% max win rate and 50\% average win rate at 2-level search depth.
		Report: \link{https://blog.qvq.moe/OmegaGomoku/Report/\%E4\%BA\%BA\%E5\%B7\%A5\%E6\%99\%BA\%E8\%83\%BD\%E8\%AF\%BE\%E7\%A8\%8B\%E6\%8A\%A5\%E5\%91\%8A.pdf}{Click Here}
	}[Python,PyTorch,DQN,Machine Learning,Deep Learning]
	\item \simpleProject[https://github.com/Oyami-Srk/Rust-shunting\_yard]{Rust-shunting\_yard}{
		Expression evaluation tool written in Rust using the Shunting Yard algorithm, equipped with function definition capabilities.
	}[Rust,Algorithm]
	\item \simpleProject[https://github.com/Oyami-Srk/rust-headless-chrome/tree/resp-handler-dereg]{Rust-headless-chrome(Code contribution)}{
		Contributed code to the Chrome DevTools Protocol implementation API in Rust, enhancing and refining certain API application methods. Additionally, fixed errors within the code generation tool (\link{https://github.com/Oyami-Srk/auto\_generate\_cdp}{\texttt{auto\_generate\_cdp}}) to solve the functionality issue of the project.
	}[Rust,Library,Chrome DevTools]
	\item \simpleProject[https://github.com/Oyami-Srk/shirodl]{ShiroDL}{
		Asynchronous concurrent small-files download library in Rust, designed for versatility and extensibility. Includes a command-line program for batch downloading small files.
	}[Rust,Library,Tool developement]
\end{itemize}

%======================================================================
\sectionTitle{Awards and Certificates}{\faTrophy}
%======================================================================
\begin{itemize}
  \item\link{https://qvq.moe:5001/public/imgs/2018-GoogleCodeIn\%E9\%AB\%98\%E4\%B8\%AD.jpg}{\texttt{Google Code-in 2017}}
	\item\link{https://qvq.moe:5001/public/imgs/2020-\%E5\%A4\%96\%E7\%A0\%94\%E7\%A4\%BE\%E5\%9B\%BD\%E6\%9D\%90\%E6\%9D\%AF\%E5\%85\%A8\%E5\%9B\%BD\%E8\%8B\%B1\%E8\%AF\%AD\%E9\%98\%85\%E8\%AF\%BB\%E5\%A4\%A7\%E8\%B5\%9B\%E5\%B1\%B1\%E4\%B8\%9C\%E7\%9C\%81\%E5\%A4\%8D\%E8\%B5\%9B-\%E4\%B8\%89\%E7\%AD\%89\%E5\%A5\%96.jpg} {\texttt{2020\enquote{FLTRP·ETIC Cup}English Reading Contest\ Shandong Provincial Final \ Third Award}}
	\item\link{https://qvq.moe:5001/public/imgs/2021-\%E4\%B8\%AD\%E5\%9B\%BD\%E5\%A4\%A7\%E5\%AD\%A6\%E7\%94\%9F\%E8\%AE\%A1\%E7\%AE\%97\%E6\%9C\%BA\%E8\%AE\%BE\%E8\%AE\%A1\%E5\%A4\%A7\%E8\%B5\%9B\%E5\%B1\%B1\%E4\%B8\%9C\%E7\%9C\%81\%E7\%BA\%A7\%E8\%B5\%9B-\%E4\%B8\%89\%E7\%AD\%89\%E5\%A5\%96.jpg} {\texttt{2021 14th Chinese Collegiate Computing Competition\ Shandong Provincial\ Third Award}}
	\item\link{https://qvq.moe:5001/public/imgs/2021-\%E5\%85\%A8\%E5\%9B\%BD\%E5\%A4\%A7\%E5\%AD\%A6\%E7\%94\%9F\%E8\%8B\%B1\%E8\%AF\%AD\%E7\%AB\%9E\%E8\%B5\%9BC\%E7\%B1\%BB-\%E4\%BA\%8C\%E7\%AD\%89\%E5\%A5\%96.jpg} {\texttt{2021 3rd National College Computer Ability Challenge\ Program design-C++Easten China area\ Excellence Award}}
	\item\link{https://qvq.moe:5001/public/imgs/2021-\%E5\%B1\%B1\%E4\%B8\%9C\%E7\%9C\%81\%E5\%A4\%A7\%E5\%AD\%A6\%E7\%94\%9F\%E7\%94\%B5\%E5\%AD\%90\%E4\%B8\%8E\%E4\%BF\%A1\%E6\%81\%AF\%E6\%8A\%80\%E6\%9C\%AF\%E5\%BA\%94\%E7\%94\%A8\%E5\%A4\%A7\%E8\%B5\%9B-\%E4\%BA\%8C\%E7\%AD\%89\%E5\%A5\%96.jpg} {\texttt{\enquote{Sai Guan Cup}8th Shandong Provincial College Students' Electronic and Information Technology Application Competition\ Second Award}}
	\item\link{https://qvq.moe:5001/public/imgs/2022-\%E4\%B8\%AD\%E5\%9B\%BD\%E5\%A4\%A7\%E5\%AD\%A6\%E7\%94\%9F\%E8\%AE\%A1\%E7\%AE\%97\%E6\%9C\%BA\%E8\%AE\%BE\%E8\%AE\%A1\%E5\%A4\%A7\%E8\%B5\%9B\%E7\%9C\%81\%E4\%BA\%8C\%E7\%AD\%89\%E5\%A5\%96.jpg} {\texttt{2022 15th Chinese Collegiate Computing Competition\ Shandong Provincial\ Second Award}}
	\item\link{https://qvq.moe:5001/public/imgs/2022-\%E6\%93\%8D\%E4\%BD\%9C\%E7\%B3\%BB\%E7\%BB\%9F\%E8\%AE\%BE\%E8\%AE\%A1\%E5\%88\%9D\%E8\%B5\%9B\%E4\%BC\%98\%E8\%83\%9C\%E5\%A5\%96.jpg} {\texttt{2022 Computer System Development Capability Competition - Kernel Design Competition\ 
Preliminary Excellence Award}}
	\item\link{https://qvq.moe:5001/public/imgs/2022-\%E8\%93\%9D\%E6\%A1\%A5\%E6\%9D\%AF\%E5\%9B\%BD\%E8\%B5\%9BB\%E7\%BB\%84-\%E4\%B8\%89\%E7\%AD\%89\%E5\%A5\%96.jpg} {\texttt{13rd Blue Bridge Cup\ Shandong Provincal C/C++\ College B Group\ First Award}}
	\item\link{https://qvq.moe:5001/public/imgs/2022-\%E8\%93\%9D\%E6\%A1\%A5\%E6\%9D\%AF\%E5\%9B\%BD\%E8\%B5\%9BB\%E7\%BB\%84-\%E4\%B8\%89\%E7\%AD\%89\%E5\%A5\%96.jpg} {\texttt{13rd Blue Bridge Cup\ National Finals C/C++\ College B Group\ Third Award}}
	\item\link{https://qvq.moe:5001/public/imgs/\%E8\%BD\%AF\%E4\%BB\%B6\%E8\%91\%97\%E4\%BD\%9C\%E6\%9D\%83\%E7\%99\%BB\%E8\%AE\%B0\%E8\%AF\%81\%E4\%B9\%A6-\%E7\%A9\%BA\%E6\%B0\%94\%E8\%B4\%A8\%E9\%87\%8F\%E6\%A3\%80\%E6\%B5\%8B\%E7\%B3\%BB\%E7\%BB\%9F.jpg} {\texttt{National Copyright Administration of China\  Computer software copyright registration certificate\ AiR Air Quality Monitoring System}}
\end{itemize}

\end{document}
